\documentclass[11pt, aspectraio=169, xcolor=table]{beamer}
\usepackage[utf8]{inputenc}
\usepackage[T1]{fontenc}
\usepackage{graphicx}
\usepackage{hyperref}
\usepackage{lmodern}
\usepackage[spanish]{babel}
\usepackage{pdfrender}
\usepackage{xcolor}
\usepackage{ragged2e}
\renewcommand{\raggedright}{\justifying}
\usepackage{smartdiagram}
\usepackage[version=4]{mhchem}
\usepackage{chemfig}


\usetheme{Berlin}

\author{Prof. Daniel Muñoz \
  \texttt{daniel.munoz3@mail.udp}
}
\title{Química}
\subtitle{Clase 3}
\date{\today}

\begin{document}

\maketitle



\section{Repaso de Lewis}

\section{Teoría de repulsión de pares de electrones de valencia (TRPEV)}

\begin{frame}{TRPEV}
	\begin{columns}
		\begin{column}{.5\textwidth}
			\begin{itemize}[<+->]
				\item La estructuras de Lewis, si bien son útiles para determinar la unión de elementos y pares electronicos libres, no nos permite, directamente saber la disposición 3D de esta.
				\item En nuestro auxilio, utilizamos la ``Teoría de repulsión de pares eletronicos enlazantes'' (TRPEV).
				\item La cual como bien dice su nombre establece que los enlaces en una molécula, buscarán adoptar la mayor distancia entre ellos en un ambiente 3D.
				\item Esta teoría también se llama teoría de Gilliespie-Nyholm en honor a sus creadores en 1957.
			\end{itemize}
		\end{column}

		\begin{column}{.5\textwidth}
			\begin{figure}[ht]
				\centering
				\includegraphics[width=0.8\textwidth]{../img/gilliespie.png}
				\caption{Creadores de la TRPEV}
			\end{figure}
		\end{column}
	\end{columns}

\end{frame}

\section{Identificación de la geometría y polaridad de las moléculas en función de los átomos que la componen según la TRPEV}

\begin{frame}{¿Cómo saber la disposición 3D de una molécula?}
	\begin{columns}
		\begin{column}{.5\textwidth}
			\begin{itemize}[<+->]
				\item Afortunadamente Gilliespie y Nyholm hicieron todos los cálculos por nosotros
				\item Por tanto solamente necesitamos conocer cómo se ordenan los ligandos alrededor del átomo central para determinar su geometría (disposición 3D)
			\end{itemize}
		\end{column}
		\begin{column}{.5\textwidth}
			\begin{figure}[ht]
				\centering
				\includegraphics[width=0.7\textwidth]{../img/molview.png}
				\caption{\url{https://molview.org}}
			\end{figure}

		\end{column}
	\end{columns}

\end{frame}

\begin{frame}{Dime tu composición y te diré tu geometría}
	\begin{columns}
		\begin{column}{.5\textwidth}
			\begin{itemize}[<+->]
				\item Para determinar la geometría de la molécula primero utilizaremos el método ABE, AXE, o ALE
				\item Seguido revisaremos, con este método según la tabla TRPEV que geometría y ángulo de enlace corresponde

			\end{itemize}


		\end{column}

		\begin{column}{.5\textwidth}
			\begin{block}{Notación ALE}
				\begin{itemize}
					\item A = átomo central
					\item L = Ligando
					\item E = par electrónico libre del átomo central
				\end{itemize}
			\end{block}

		\end{column}
	\end{columns}
\end{frame}

\begin{frame}{TRPEV}

	\begin{figure}[ht]
		\centering
		\includegraphics[width=0.4\textwidth]{../img/trpev.png}
		\caption{\url{https://es.wikipedia.org/wiki/TRePEV}}
	\end{figure}
\end{frame}

\begin{frame}{Polaridad}
	\begin{columns}
		\begin{column}{.5\textwidth}
			La polaridad de una molécula dependerá de:
			\begin{itemize}[<+->]
				\item Su distribución de carga, si es homogénea, será \textit{apolar}
				\item En caso contrarío será \textit{polar}
			\end{itemize}
		\end{column}

		\begin{column}{.5\textwidth}
			\begin{figure}[ht]
				\centering
				\includegraphics<1>[width=0.6\textwidth]{../img/aceite}
				\includegraphics<2>[width=0.6\textwidth]{../img/agua}
			\end{figure}

		\end{column}
	\end{columns}

\end{frame}

\begin{frame}{Análisis de la distribución de carga}

\end{frame}

\section{Predicción del tipo de interacciones intermoleculares}

\begin{frame}{Tipos de interacciones intermoleculares}
  \begin{columns}
    \begin{column}{.5\textwidth}
 	\begin{itemize}[<+->]
		\item ion-ion
		\item ion-diper
    \item Fuerzas de Van der Waals
    \begin{itemize}
		\item diper-diper
		\item diper-dipin
		\item Fuerzas de dispersión (London)
	\end{itemize}
    \end{itemize}
    \end{column}
    \begin{column}{.5\textwidth}
    \end{column}
  \end{columns}
\end{frame}

\begin{frame}{Esto, cómo afecta la formación de las disoluciones en:}
  \begin{itemize}[<+->]
	\item la formación de soluciones
	\item Propiedades físicas.
	 \begin{itemize}[<+->]
	\item Solubilidad
	\item Punto de fusión
	\item Conductividad eléctrica.
	 \end{itemize}
 \end{itemize}
\end{frame}
\end{document}
