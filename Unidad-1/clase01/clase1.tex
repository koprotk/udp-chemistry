% Created 2025-03-09 dom 18:01
% Intended LaTeX compiler: pdflatex
\documentclass[11pt, aspectratio=169, xcolor=table]{beamer}
\usepackage[utf8]{inputenc}
\usepackage[T1]{fontenc}
\usepackage{graphicx}
\usepackage{longtable}
\usepackage{wrapfig}
\usepackage{rotating}
\usepackage[normalem]{ulem}
\usepackage{amsmath}
\usepackage{amssymb}
\usepackage{capt-of}
\usepackage{hyperref}
\usepackage[spanish]{babel}
\usepackage[version=4]{mhchem}
\usepackage{ragged2e}
\renewcommand{\raggedright}{\justifying}
\usetheme{Berlin}
\author{Prof. Daniel Muñoz}
\date{\textit{<2025-03-06 jue>}}
\title{Química}
\subtitle{Clase 1.}
\titlegraphic{\includegraphics[width=4cm]{../img/udplogo}}
\hypersetup{
	pdfauthor={Prof. Daniel Muñoz},
	pdftitle={Química},
	pdfkeywords={},
	pdfsubject={},
	pdfcreator={Emacs 29.4 (Org mode 9.7.22)},
	pdflang={Spanish}}
\begin{document}

\maketitle
\begin{frame}{Outline}
	\tableofcontents
\end{frame}

\section{Presentación}
\label{sec:org93e1c60}
\begin{frame}[label={sec:orgdde78ec}]{Acerca del Profesor}
	\begin{columns}
		\begin{column}{0.5\columnwidth}
			\footnotesize
			\begin{itemize}[<+->]
				\item Daniel E. Muñoz Masson daniel.munoz3@mail.udp.cl
				\item Profesor de Química. Uchile
				\item Mg. en educación c/m Evaluación de Aprendizajes
				\item Mg. En Ciencias Químicas c/m Fisicoquímica. Uchile
				\item Analista Data Science con Python. Corfo
				\item Analista QA en Automatización de Pruebas. Corfo
				\item Experiencia como: Jefe de Proyectos, Docente, Automatizador, Académico-Investigador.
			\end{itemize}
		\end{column}
		\begin{column}{0.5\columnwidth}
			\begin{figure}[htbp]
				\centering
				\includegraphics[width=0.8\textwidth]{../img/yo.jpg}
				\caption{\emph{Le Fils de l'Homme} 1964 René Magritte}
			\end{figure}
		\end{column}
	\end{columns}
\end{frame}
\begin{frame}[label={sec:org1905b57}]{Acerca del Curso}
	\begin{columns}
		\begin{column}{0.5\columnwidth}
			\begin{itemize}
				\item Clases:
				      \begin{itemize}
					      \item I: \textit{<2025-03-06 jue>}
					      \item F: \textit{<2025-06-27 vie>}
				      \end{itemize}
				\item Aprobación:
				      \begin{itemize}
					      \item Promedio General => 4.0
				      \end{itemize}
				\item Eximición:
				      \begin{itemize}
					      \item PG => 5.0
					      \item Ninguna PS < 4.0
					      \item Haber rendido todas las S y \alert{TI}
				      \end{itemize}
			\end{itemize}
		\end{column}
		\begin{column}{0.5\columnwidth}
			\begin{figure}[htbp]
				\centering
				\includegraphics[width=0.6\textwidth]{../img/qca1.jpg}
				\caption{Orbitales del molecular}
			\end{figure}
		\end{column}
	\end{columns}
\end{frame}
\begin{frame}[label={sec:org17cacc8}]{Evaluaciones Sumativas (con nota)}
	\begin{columns}
		\begin{column}{0.5\columnwidth}
			\begin{itemize}
				\item Pruebas solemnes (S1 y S2) = 60\%
				\item Controles (C1, C2, C3, C4) = 10\%
				\item Trabajo de Integración (Q, OG, VID) = 30\%
				\item NP = C*10\% + S1*30\% + S2*30\% + TI*30\%
				\item Examen (Contenido de S1 y S2), reemplaza la peor S en la NF
				\item NF = NP*70\% + E*30\%
			\end{itemize}
		\end{column}
		\begin{column}{0.5\columnwidth}
			\begin{figure}[htbp]
				\centering
				\includegraphics[width=0.7\textwidth]{../img/chlorophyl.png}
				\caption{Clorofila y Hemoglobina}
			\end{figure}
		\end{column}
	\end{columns}
\end{frame}
\begin{frame}[label={sec:orgcb0115c}]{Trabajo de integración}
	\begin{columns}
		\begin{column}{0.5\columnwidth}
			\begin{itemize}
				\item Trabajo \alert{Grupal} estudio de manuscrito de algún \emph{tema dado}
				\item Se divide en:
				      \begin{itemize}
					      \item Cuestionario (Q): Respuesta a un cuestionario dividido en dos partes. (\emph{Heteroevaluado})
					      \item Organizador Gráfico (OG): Herramienta visual que sintetiza el \emph{tema dado}. (\emph{Autoevaluado})
					      \item Presentación Audiovisual (VID): Video que presenta sintéticamente el trabajo de integración (\emph{Coevaluado})
				      \end{itemize}
			\end{itemize}
		\end{column}
		\begin{column}{0.5\columnwidth}
			\begin{figure}[htbp]
				\centering
				\includegraphics[width=0.7\textwidth]{../img/protein.jpg}
				\caption{Estructura cuaternaria de las proteínas}
			\end{figure}
		\end{column}
	\end{columns}
\end{frame}
\begin{frame}[label={sec:org892e306}]{Bibliografía del curso}
	\begin{columns}
		\begin{column}{0.3\columnwidth}
			\begin{block}{Básica}
				\begin{thebibliography}{Chang, 2010}
					\setbeamertemplate{bibliography item}[book]
					\bibitem[Chang, 2011]{Chang2011}
					Chang, Raymond
					\newblock \emph{Fundamentos de Química}
					\newblock McGraw Hill, 2011

					\setbeamertemplate{biblography item}[book]
					\bibitem[Chang, 2010]{Chang2010}
					Chang, Raymond
					\newblock \emph{Química}, 10° Edición
					\newblock McGraw Hill, 2010
				\end{thebibliography}
			\end{block}
		\end{column}
		\begin{column}{0.7\columnwidth}
			\begin{block}{Complementaria}
				\begin{thebibliography}{Serway, 2008}
					\footnotesize
					\setbeamertemplate{bibliography item}[book]
					\bibitem[Brown, 2009]{Brown2009}
					Brown, T.L; LeMay, H.E; Bursten, B.E y col.
					\newblock \emph{Química: La Ciencia Central}
					\newblock Pearson, 2009

					\setbeamertemplate{biblography item}[book]
					\bibitem[Petrucci, 2011]{Petrucci2010}
					Petrucci, R.H; Herring, F.G; Madura, J.D y Bissonnette, C.
					\newblock \emph{Química General: Principios y Aplicaciones Modernas} 10° Edición
					\newblock Pearson, 2011
					\setbeamertemplate{bibliography item}[book]
					\bibitem[Serway, 2008]{Serway2008}
					Serway, R. y Jewett, J.
					\newblock \emph{Física para Ciencias e Ingeniería}, 7° Edición
					\newblock Brooks/Cole, 2008
				\end{thebibliography}
			\end{block}
		\end{column}
	\end{columns}
\end{frame}
\begin{frame}[label={sec:org73574b3}]{Ahora preséntese usted\ldots{}}
	\begin{center}
		\includegraphics[height=5cm]{../img/uncleSam.jpg}
	\end{center}
\end{frame}
\section{Configuración electrónica}
\label{sec:org4302b14}
\begin{frame}[label={sec:org31f4d27}]{Historia del átomo y el elemento: Demócrito/Aristóteles}
	\begin{columns}
		\begin{column}{0.5\columnwidth}
			\begin{itemize}[<+->]
				\item Desde el inicio de los tiempos que la humanidad se ha preguntado \emph{de qué están hecha las cosas}.
				\item Los primeros avances se registran en la Grecia clásica (400 a.C.) Demócrito de Abdera postuló que las \emph{cosas} está hechas de objetos indivisibles llamados \(\alpha\) (a = sin) y \(\tau\)\textit{o}\(\mu\)\textit{o}\(\nu\) (tomo = división).
				\item Esta idea fue sometida a la crítica de Aristóteles (350 a.C.), siendo las ideas de este último las que prevalecieron hasta el SXVIII.
			\end{itemize}
		\end{column}
		\begin{column}{0.5\columnwidth}
			\only<1-2>{
				\begin{figure}[htbp]
					\centering
					\includegraphics[width=0.5\textwidth]{../img/democrito.jpg}
					\caption{Demócrito de Abdera 460 a.C. - 370 a.C.}
				\end{figure}
			}

			\only<3>{
				\begin{figure}[htbp]
					\centering
					\includegraphics[width=0.5\textwidth]{../img/aristoteles.jpg}
					\caption{Aristóteles 384 a.C. - 322 a.C.}
				\end{figure}
			}
		\end{column}
	\end{columns}
\end{frame}
\begin{frame}[label={sec:orge58fed4}]{John Dalton y Michael Faraday}
	\begin{columns}
		\begin{column}{0.5\columnwidth}
			\footnotesize
			\begin{itemize}
				\item John Dalton científico-profesor inglés siguió la posta del desarrollo de la teoría atómica.
				\item En 1808 en \guillemotleft{}Nuevo Sistema de filosofía química\guillemotright{} menciona: la \emph{materia se compone de partículas atómicas; los átomos de un mismo \guillemotleft{}elemento\guillemotright{} son iguales en su peso y cualidad. Los compuestos nacen por la unión de átomos de dos o más elementos diferentes}.
				\item En 1883 Michael Faraday otro científico inglés, descubrió que el flujo de corriente eléctrica produce cambios, por tanto sugiere que los átomos deben tener una estructura eléctrica.
			\end{itemize}
		\end{column}
		\begin{column}{0.5\columnwidth}
			\only<1>{
				\begin{figure}[htbp]
					\centering
					\includegraphics[width=0.5\textwidth]{../img/dalton.jpg}
					\caption{John Dalton 1766 - 1844}
				\end{figure}
			}

			\only<2>{
				\begin{figure}[htbp]
					\centering
					\includegraphics[height=0.6\textheight]{../img/dalton-atomic.jpg}
					\caption{Teoría Atómica de Dalton}
				\end{figure}
			}

			\only<3>{
				\begin{figure}[htbp]
					\centering
					\includegraphics[width=0.5\textwidth]{../img/faraday.jpg}
					\caption{Machael Faraday 1791 - 1867}
				\end{figure}
			}
		\end{column}
	\end{columns}
\end{frame}
\begin{frame}[label={sec:org29bbfba}]{J.J. Thomson}
	\begin{columns}
		\begin{column}{0.5\columnwidth}
			\begin{itemize}[<+->]
				\item Joseph John Thomson, científico inglés en 1906 a partir del experimento de los \guillemotleft{}rayos catódicos\guillemotright{}, logra desarrollar el primer modelo atómico con estructura interna a partir de datos experimentales.
				\item Modelo atómico de Thomson: una base positiva con incrustaciones negativas de partículas subatómicas las cuales nombró como \emph{electrones}.
			\end{itemize}
		\end{column}
		\begin{column}{0.5\columnwidth}
			\only<1>{
				\begin{figure}[htbp]
					\centering
					\includegraphics[width=0.5\textwidth]{../img/thomson.jpg}
					\caption{J.J. Thomson 1856 - 1940}
				\end{figure}
			}

			\only<2>{
				\begin{figure}[htbp]
					\centering
					\includegraphics[width=0.7\textwidth]{../img/rayoscatodicos.jpg}
					\caption{Máquina de rayos catódicos}
				\end{figure}
			}

			\only<3>{
				\begin{figure}[htbp]
					\centering
					\includegraphics[width=0.6\textwidth]{../img/thomsonmodel.jpeg}
					\caption{Modelo atómico de Thomson}
				\end{figure}
			}
		\end{column}
	\end{columns}
\end{frame}
\begin{frame}[label={sec:orge0172ec}]{Ernest Rutherford}
	\begin{columns}
		\begin{column}{0.5\columnwidth}
			\begin{itemize}[<+->]
				\item Ernest Rutherford científico inglés en 1911 a partir del experimento de la \guillemotleft{}lámina de oro\guillemotright{} logra descubrir que el átomo en su mayoría es espacio vació.
				\item Modelo atómico de Rutherford (planetario): un núcleo positivo con electrones orbitando alrededor del núcleo.
				\item Pero había un problema con este modelo \ldots{}
			\end{itemize}
		\end{column}
		\begin{column}{0.5\columnwidth}
			\only<1>{
				\begin{figure}[htbp]
					\centering
					\includegraphics[width=0.5\textwidth]{../img/rutherford.jpg}
					\caption{Ernest Rutherford 1871 - 1937}
				\end{figure}
			}

			\only<2>{
				\begin{figure}[htbp]
					\centering
					\includegraphics[width=0.7\textwidth]{../img/goldslide.jpeg}
					\caption{Experimento de la lámina de oro.}
				\end{figure}
			}

			\only<3>{
				\begin{figure}[htbp]
					\centering
					\includegraphics[width=0.6\textwidth]{../img/rutherfordmodel.jpeg}
					\caption{Modelo atómico de Rutherford}
				\end{figure}
			}
		\end{column}
	\end{columns}
\end{frame}
\begin{frame}[label={sec:orgac9dc89}]{El salto a la mecánica cuántica y la pérdida de las esferas duras.}
	\begin{columns}
		\begin{column}{0.5\columnwidth}
			\footnotesize
			\begin{itemize}[<+->]
				\item Después de Bohr, ingentes científicos hicieron aportes inconmensurables al entendimiento del átomo y del universo subatómico, entre los exponentes más destacados: De Broglie, E. Schrödinger, W. Heisenberg, J. Slater, P. Dirac, W. Pauli, entre otros.
				\item Esto avances nos llevaron a una interpretación \emph{probabilista} de la \emph{realidad} en contraste con la clásica \emph{causalidad} que imperaba en la física clásica.
			\end{itemize}
		\end{column}
		\begin{column}{0.5\columnwidth}
			\only<1>{
				\begin{figure}[htbp]
					\centering
					\includegraphics[height=0.6\textheight]{../img/proceres.png}
					\caption{Collage, diferentes científicos}
				\end{figure}
			}

			\only<2>{
				\huge
				\begin{itemize}
					\item \texttimes{} \(A \to B\)
					\item \(\checkmark\) \(\Psi\)
				\end{itemize}
			}
		\end{column}
	\end{columns}
\end{frame}
\begin{frame}[label={sec:orgac0438e}]{Niels Bohr y el advenimiento de la mecánica cuántica.}
	\begin{columns}
		\begin{column}{0.5\columnwidth}
			\footnotesize
			\begin{itemize}[<+->]
				\item Niels Bohr, científico danés en 1913 profundiza en el modelo atómica de Rutherford, integrando los incipientes descubrimientos de una nueva física, la física cuántica.
				\item De los trabajos sobre el modelo atómico de Rutherford, introduce le número cuántico \guillemotleft{}n\guillemotright{} el cuál representaría la órbita del electrón, además concluyendo que no todos los electrones circulan por todas las orbitas, estableciendo que estos saltan de una a otra emitiendo energía.
				\item Esta interpretación permitió explicar el fenómeno de los espectros atómicos.
			\end{itemize}
		\end{column}
		\begin{column}{0.5\columnwidth}
			\only<1>{
				\begin{figure}[htbp]
					\centering
					\includegraphics[height=0.6\textheight]{../img/bohr.jpg}
					\caption{Niels Bohr 1885 - 1962}
				\end{figure}
			}

			\only<2>{
				\begin{figure}[htbp]
					\centering
					\includegraphics[width=0.7\textwidth]{../img/bohrmodel.png}
					\caption{Modelo atómico de Bohr.}
				\end{figure}
			}

			\only<3>{
				\begin{figure}[htbp]
					\centering
					\includegraphics[width=0.7\textwidth]{../img/espectro.jpeg}
					\caption{Espectros}
				\end{figure}
			}
		\end{column}
	\end{columns}
\end{frame}
\begin{frame}[label={sec:org67fbb82}]{Entonces, ¿Como describimos un átomo?}
	\begin{columns}
		\begin{column}{0.5\columnwidth}
			\begin{itemize}[<+->]
				\item Un átomo posee:
				      \begin{itemize}
					      \item Número atómico; \(Z = \sum p^+\)
					      \item Número másico; \(A = Z + \sum n^0\)
					      \item Simbolo atómico
					      \item Carga; \(Q = Z - \sum e^-\)
				      \end{itemize}
			\end{itemize}
		\end{column}
		\begin{column}{0.5\columnwidth}
			\huge
			\centering
			\ce{^{\textbf<3>{227}}_{\textbf<2>{90}}{\textbf<4>{Th}}^{\textbf<5>{+}}}
			\vspace{1cm}

			\pause

			\small
			\(\sum\) p\textsuperscript{+} = Z = 90; \(\sum\) n\textsuperscript{0} = 137; \(\sum\) e\textsuperscript{-} = 89
		\end{column}
	\end{columns}
\end{frame}
\begin{frame}[label={sec:orgbc65b59}]{Configuración electrónica}
	\begin{columns}
		\begin{column}{0.5\columnwidth}
			\begin{itemize}[<+->]
				\item La configuración electrónica es la forma en que describimos los electrones de un elemento.
				\item Esta caracterización de los electrones de un elemento se logra mediante el uso de los llamados \guillemotleft{}números cuánticos\guillemotright{}:
				      \begin{itemize}
					      \item Número cuántico principal \alert{n}
					      \item Número cuántico secundario \alert{l}
					      \item Número cuántico magnético \alert{m\textsubscript{l}}
					      \item Número cuántico de espín \alert{m\textsubscript{s}}
				      \end{itemize}
			\end{itemize}
		\end{column}
		\begin{column}{0.5\columnwidth}
			\begin{center}
				\includegraphics<1-2>[width=.9\linewidth]{../img/settings.jpeg}
			\end{center}

			\begin{itemize}
				\item <3-> Adquiere valores desde \(1, 2, 3 ... \infty\)
				\item <4-> Adquiere valores desde \(0, 1, 2 ... n-1\)
				\item <5-> Adquiere valores desde: \(-l, -l+1, -l+2 ... , 0, 1, 2, ... ,+l-1, +l\)
				\item <6-> Adquiere valores de: \(+\frac{1}{2}, -\frac{1}{2}\)
			\end{itemize}
		\end{column}
	\end{columns}
\end{frame}
\begin{frame}[label={sec:orgc111f66}]{Orbitales atómicos}
	\begin{columns}
		\begin{column}{0.5\columnwidth}
			\begin{itemize}[<+->]
				\item Cada valor de l se le asigna una letra:
				\item Cada combinación de los tres números cuánticos: n, l y m\textsubscript{l} se les llama \emph{orbital atómico}.
				\item Para combinarlos: se utiliza la notación \(nl_{m_l}\)
			\end{itemize}
		\end{column}
		\begin{column}{0.5\columnwidth}
			\begin{block}{}
				\only<1>{
					\begin{center}
						\begin{tabular}{rl}
							l & eq \\
							\hline
							0 & s  \\
							1 & p  \\
							2 & d  \\
							3 & f  \\
							4 & g  \\
						\end{tabular}
					\end{center}
				}

				\only<2->{
					\begin{center}
						\begin{tabular}{rrrl}
							n & l & m\textsubscript{l} & \(\psi\)             \\
							\hline
							1 & 0 & 0                  & 1s                   \\
							2 & 0 & 0                  & 2s                   \\
							2 & 1 & -1                 & 2p\textsubscript{-1} \\
							2 & 1 & 0                  & 2p\textsubscript{0}  \\
							2 & 1 & +1                 & 2p\textsubscript{+1} \\
						\end{tabular}
					\end{center}

				}
			\end{block}
		\end{column}
	\end{columns}
\end{frame}
\begin{frame}[label={sec:orgb34869a}]{¿Cómo se construye una CE a partir de los electrones de un átomo?}
	\begin{columns}
		\begin{column}{0.5\columnwidth}
			\begin{itemize}[<+->]
				\item Se deben seguir ciertas reglas:
				      \begin{itemize}
					      \item \emph{Principio de mínima energía}: Los electrones inician con orbitales de menor energía (\(n+l\)) hacia otros de mayor energía
					      \item \emph{Principio de exclusión de Pauli}: Cada orbital acepta, como máximo, \alert{dos} electrones.
					      \item \emph{Regla de Hund}: Los electrones van adquiriendo diferentes valores de m\textsubscript{l} para el mismo l antes de repetir.
				      \end{itemize}
			\end{itemize}
		\end{column}
		\begin{column}{0.5\columnwidth}
			\begin{figure}[htbp]
				\centering
				\includegraphics[width=0.6\textwidth]{../img/dmoeller.jpeg}
				\caption{Diagrama de Moeller}
			\end{figure}
		\end{column}
	\end{columns}
\end{frame}
\begin{frame}[label={sec:org67836c0}]{Ejercicios:}
	\begin{block}{Ejercicio 1}
		Escriba la configuración electrónica completa y los 4 números cuánticos del último electrón para los siguientes átomos:
		\begin{itemize}
			\item \ce{_9F}
			\item \ce{_2He+}
			\item \ce{_6C}
		\end{itemize}
	\end{block}
\end{frame}

\begin{frame}[label={sec:orgd17db53}]{Bibliografía}
	\begin{thebibliography}{Scerri, 2007}
		\footnotesize
		\setbeamertemplate{bibliography item}[online]
		\bibitem[Labnews, 2019]{Labnews2019}
		Laboratory news
		\newblock \emph{Alternative periodic tables}
		\newblock \url{https://www.labnews.co.uk/article/2029799/alternative-periodic-tables}

		\setbeamertemplate{bibliography item}[online]
		\bibitem[Lifeder, 2022]{Lifeder2022}
		Lifeder
		\newblock \emph{Tríadas de Döbereiner}
		\newblock \url{https://www.lifeder.com/triadas-de-dobereiner/}

		\setbeamertemplate{biblography item}[online]
		\bibitem[Energía Nuclear, 2023]{Nuclear2023}
		Energía Nuclear
		\newblock \emph{Ley de las Octavas de Newlands}
		\newblock \url{https://energia-nuclear.net/quimica/tabla-periodica/linea-del-tiempo/ley-de-las-octavas}
		\setbeamertemplate{bibliography item}[book]
		\bibitem[Scerri, 2007]{Scerri2007}
		Scerri, Eric.
		\newblock \emph{The Periodic Table: It's Story and Its Significance}
		\newblock Oxford University Press, 2007
	\end{thebibliography}
\end{frame}
\end{document}
