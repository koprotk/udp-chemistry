\documentclass[11pt, aspectratio=169, xcolor=table]{beamer}
\usepackage[utf8]{inputenc}
\usepackage[T1]{fontenc}
\usepackage{graphicx}
\usepackage{hyperref}
\usepackage{lmodern}
\usepackage[spanish]{babel}
\usepackage{pdfrender}
\usepackage{xcolor}
\usepackage{ragged2e}
\usepackage[version=4]{mhchem}
\renewcommand{\raggedright}{\justifying}
\usepackage{smartdiagram}
\usetheme{Berlin}

\author{Prof. Daniel Muñoz \\
	\texttt{daniel.munoz3@mail.udp}
}
\title{Química}
\subtitle{Clase 4}
\date{16 de marzo de 2025}

\begin{document}

\maketitle

\section{Interacciones Intermoleculares y formación de disoluciones}

\begin{frame}{¿Cómo ocurre la disolución?}
	\begin{columns}
		\begin{column}{.5\textwidth}
			\begin{itemize}[<+->]
				\item Proceso de solvatación
				\item Fuerzas soluto-solvente vs. fuerzas intramoleculares
				\item Ejemplo: NaCl en agua
			\end{itemize}
		\end{column}

		\begin{column}{.5\textwidth}
			\begin{figure}[ht]
				\centering
				\includegraphics[width=\textwidth]{../img/water-salt.jpg}
				\caption{Representación de NaCl disolviéndose en agua (esferas de \ce{Na+} y \ce{Cl-} rodeadas por \ce{H2O})}
			\end{figure}
		\end{column}
	\end{columns}
\end{frame}

\begin{frame}{Lo semejante disuele lo semejante}
	\begin{columns}
		\begin{column}{.5\textwidth}
			\begin{itemize}[<+->]
				\item Disolución de sustancias polares en solventes polares
				\item Disolución de sustancias apolares en solventes apolares
				\item Ejemplo: NaCl en agua vs. aceite en gasolina
			\end{itemize}
		\end{column}
		\begin{column}{.5\textwidth}
			\begin{figure}[ht]
				\centering
				\includegraphics[width=0.8\textwidth]{../img/vitc-sol.jpg}
				\caption{Comparación de solubilidad de la Vitamina C en diferentes solventes (menos RA más soluble) \cite{Aristizabal2016}}
			\end{figure}
		\end{column}
	\end{columns}
\end{frame}

\begin{frame}{Factores que afectan la solubilidad}
	\begin{columns}
		\begin{column}{.5\textwidth}
			\begin{itemize}[<+->]
				\item Naturaleza del soluto y solvente
				\item Temperatura:
				      \begin{itemize}
					      \item Aumento en solubilidad de sólidos en líquidos
					      \item Disminución en solubilidad de gases en líquidos
				      \end{itemize}
				\item Presión (Ley de Henry)
			\end{itemize}
		\end{column}

		\begin{column}{.5\textwidth}
			\begin{figure}[ht]
				\centering
				\includegraphics<2-4>[width=0.6\textwidth]{../img/temp-solv.jpeg}
				\includegraphics<5>[width=0.8\textwidth]{../img/henry-law.png}
				\only<2-4>{\caption{Disolver azúcar: agua fría vs caliente}}
				\only<5>{\caption{Proceso de abrir una geaseosa}}
			\end{figure}
		\end{column}
	\end{columns}

\end{frame}

\section{Influencia de las interacciones en las propiedades físicas}

\begin{frame}{Punto de fusión y ebullición}
	\begin{columns}
		\begin{column}{.5\textwidth}
			\begin{itemize}[<+->]
				\item Relación con la fuerza de las interacciones intermoleculares
				\item Comparación entre sustancias con distintos tipos de fuerza
				\item Ejemplo: agua vs. etanol vs. butano
			\end{itemize}
		\end{column}

		\begin{column}{.5\textwidth}
			\begin{figure}[ht]
				\centering
				% This LaTeX table template is generated by emacs 29.4
				\only<1-2>{
					\begin{tabular}{|l|l|}
						\hline
						Sustancia & \ce{T_{eb}} °C \\
						\hline
						Agua      & 100            \\
						\hline
						Etanol    & 78             \\
						\hline
						Sal       & 1465           \\
						\hline
					\end{tabular}
					\caption{Temperatura de ebullición de diferentes sustancias}
					\includegraphics<3>[width=\textwidth]{../img/ff-inter.png}
				}

			\end{figure}

		\end{column}
	\end{columns}
\end{frame}

\begin{frame}{Solubilidad en diferentes solventes}
	\begin{columns}
		\begin{column}{.5\textwidth}
			\begin{itemize}[<+->]
				\item Diferencias en solubilidad según el tipo de interacción
				\item Ejemplo: NaCl soluble en agua pero insoluble en hexano
				\item Importancia en la industria farmacéutica
			\end{itemize}
		\end{column}

		\begin{column}{.5\textwidth}
			\begin{table}[ht]
				\centering
				% This LaTeX table template is generated by emacs 29.4
				\begin{tabular}{|l|l|}
					\hline
					Sustancia  & Solubilidad \\
					\hline
					NaCl       & 36          \\
					\hline
					KCl        & 34          \\
					\hline
					\ce{NaNO3} & 88          \\
					\hline
					Azúcar     & 203         \\
					\hline
				\end{tabular}
				\caption{Tabla de solubilidad cada 100g de agua a 20°C}
			\end{table}
		\end{column}
	\end{columns}
\end{frame}

\begin{frame}{Conductividad eléctrica en soluciones}
	\begin{columns}
		\begin{column}{.5\textwidth}
			\begin{itemize}[<+->]
				\item Diferencia entre solutos iónicos y covalentes.
				\item Ejemplo: NaCl en agua vs. azúcar en agua.
				\item Importancia en electrolitos biológicos.
			\end{itemize}
		\end{column}

		\begin{column}{.5\textwidth}
			\begin{figure}[ht]
				\centering
				\includegraphics<1-2>[width=\textwidth]{../img/disolucion.png}
				\only<1-2>{\caption{Representación iones en disolución}}
				\includegraphics<3>[width=\textwidth]{../img/electrolito-no-ele.jpg}
				\only<3>{\caption{Electrolito vs No electrolitos}}
			\end{figure}
		\end{column}
	\end{columns}
\end{frame}

\section{Relación entre estructura y funcionalidad en sustancias cotidianas}

\begin{frame}{Aplicaciones en la vida diaria}
	\begin{columns}
		\begin{column}{.5\textwidth}
			\begin{itemize}[<+->]
				\item ¿Por qué los aceites no se mezclan con el agua?
				\item ¿Por qué el alcohol se evapora rápido?
				\item ¿Por qué el hielo flota en agua?
			\end{itemize}
		\end{column}
		\begin{column}{.5\textwidth}
			\begin{figure}[ht]
				\centering
				\includegraphics<1>[width=\textwidth]{../img/inmiscible.jpeg}
				\includegraphics<2>[width=\textwidth]{../img/boiling-points.png}
				\includegraphics<3>[width=\textwidth]{../img/water-ice.jpg}
			\end{figure}
		\end{column}
	\end{columns}
\end{frame}

\begin{frame}{Materiales y tecnología}
	\begin{columns}
		\begin{column}{.5\textwidth}
			\begin{itemize}[<+->]
				\item Polímeros hidrofóbicos en ropa impermeable
				\item Solubilidad en medicamentos
				\item Electrólitos en baterías
			\end{itemize}
		\end{column}
		\begin{column}{.5\textwidth}
			\begin{figure}[ht]
				\centering
				\includegraphics<1>[width=0.8\textwidth]{../img/hidrophobic.jpeg}
				\includegraphics<2>[width=0.8\textwidth]{../img/drug-transport.png}
				\includegraphics<3>[width=0.6\textwidth]{../img/battery-flow.jpeg}
			\end{figure}
		\end{column}
	\end{columns}
\end{frame}

\section{Cierre y Reflexión}

\begin{frame}{Resumen general}
	\begin{itemize}[<+->]
		\item Tipos de interacciones intermoleculares
		\item Influencia en solubilidad y propiedades físicas
		\item Aplicaciones en la vida cotidiana
	\end{itemize}
\end{frame}

\begin{frame}{Preguntas y discusión}
	\begin{columns}
		\begin{column}{.5\textwidth}
			\begin{itemize}[<+->]
				\item Espacio para resolver dudas
				\item Pregunta final: ¿Cómo podrías aplicar estos conceptos en tu campo de estudio o en tu vida diaria?
			\end{itemize}
		\end{column}

		\begin{column}{.5\textwidth}
			\begin{figure}[ht]
				\centering
				\includegraphics[width=0.8\textwidth]{../img/discussion.jpg}
			\end{figure}

		\end{column}
	\end{columns}

\end{frame}

\begin{frame}{Bibliografía}
	\begin{thebibliography}{Dyna, 2016}
		\footnotesize
		\setbeamertemplate{bibliography item}[article]
		\bibitem[Aristizabal, 2016]{Aristizabal2016}
		Dyna
		\newblock \emph{Determinación numérica de la solubilidad de la vitamina C en diferentes solventes, para la extracción selectiva o para la incorporación en formulaciones orientadas al cuidado, bienestar y salud de la piel}
		\newblock \url{https://doi.org/10.15446/dyna.v83n199.54828}
	\end{thebibliography}
\end{frame}

\end{document}
